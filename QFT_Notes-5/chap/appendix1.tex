\chapter*{第一章附录}
\section*{不存在二阶矩阵满足Dirac方程的要求}
这一点其实非常简单。作为一个必要条件,首先我们试图找到足够多的满足$\textbf{A}^{2}=1$的矩阵。注意到对于任一$2\times2$矩阵
\begin{equation}
\textbf{A}=
    \left(                 
  \begin{array}{cc}   
    a & b \\  
    c & d \\  
  \end{array}
\right)        
\end{equation}
若其满足$\textbf{A}^{2}=1$,我们会得到
\begin{equation}
\label{con1}
\left\{
        \begin{array}{llll}
            a^{2}+bc=1 \\
            cb+d^{2}=1 \\
            b(a+d)=0 \\
            c(a+d)=0
        \end{array}
    \right.
\end{equation}
之后是一些简单的观察及分类讨论。

如果我们有$a+d\neq0$,则$b=c=0$,从而我们有
\begin{equation}
\left\{
        \begin{array}{lll}
            |a|=|d|\\
            a+d=0 \\
            b=c=0
        \end{array}
    \right.
\end{equation}
从而此时只有$\textbf{B}_{1}=\lambda\textbf{I}$满足条件,其中$\lambda$为任意复数。由于单位阵与任意非零矩阵的反对易关系非零,从(\ref{requsetofDirac})中我们可以看到,此时不存在满足条件的解。

如果$a+d=0$,则(\ref{con1})变为
\begin{equation}
\left\{
        \begin{array}{ll}
            a^{2}+bc=1 \\
            a=-d
        \end{array}
    \right.
\end{equation}
作为复数域上的矩阵,
\begin{equation}
\textbf{A}=
    \left(                 
  \begin{array}{cc}   
    a & b \\  
    c & -a \\  
  \end{array}
\right)        
\end{equation}
有3个独立参数,而从(\ref{requsetofDirac})中我们可以得到$\pmb{\alpha}^{j}$与$\pmb{\beta}$是线性无关的(令$a_{i}\pmb{\alpha}^{i}+b\pmb{\beta}=0$,然后依次计算其与各个矩阵的反对易关系并利用(\ref{requsetofDirac})),从而至少应该存在四个线性独立的矩阵,因此在2维情况下找不到满足二阶矩阵满足Dirac方程的要求。